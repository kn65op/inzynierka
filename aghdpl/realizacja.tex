\chapter{Realizacja}
\label{cha:realizacja}
W tym rozdziale przedstawiono sposób realizacji systemu do kontroli dostępu używającego tęczówki jako elementu na podstawie, którego następuje rozpoznanie. System składa się ze stanowiska do akwizycji obrazu tęczówki oka oraz aplikacji generującej kod tęczówki oraz porównującej kody ze sobą. Aplikacja składa się z kliku części: wykrycie źrenicy, wykrycie granic tęczówki, tworzenie kodu tęczówki oraz ewentualnie porównania wygenerowane kodu z już istniejącym.

\section{Stanowisko}
\label{sec:stanowisko}
Tworząc stanowisko do akwizycji obrazu tęczówki oka należy przezwyciężyć wiele problemów. Na początek należy zastanowić się na kamerą oraz obiektywem. Wybór kamery był ograniczony zasobów laboratorium, w którym było tworzone stanowisko. Do realizacji wybrano kamerę !!!!TU WSTAWIĆ NAZWĘ!!!!. Do wybranej kamery należało dobrać obiektyw. Ponieważ naszym celem było otrzymanie ostrego obrazu z bliskiej odległości oraz dużej głębi ostrości należało wybrać obiektyw o możliwie dużej ogniskowej. Wybrano najlepszy dostępny obiektyw !!!!!TU WSTAWIĆ NAZWĘ!!!!. Kolejnym problemem, który należało pokonać były odblaski na powierzchni oka. Odblaski takie powstają w wyniku odbicia się światła pochodzącego od światła zewnętrznego oraz od światła żarówek i jarzeniówek świecących w pomieszczeniu. Głównym problemem były odblaski powstające na tęczówce oraz źrenicy, ponieważ te elementy obrazu nas interesowały. Najprostszym sposobem pozbycia się takich odblasków jest zasłonięcie drogi promieni świetlnych od źródła do oka. Zrealizowaliśmy to za pomocą kartonowego pudła szczelnie, w którym umieściliśmy kamerę. Osoba badana umieszcza głowę na granicy pudła, co pozwala na uniknięcie odblasków. Dodatkowo głowa badanej osoby jest umieszczona na stojaku, którego zadaniem jest unieruchomienie głowy, dzięki czemu uchwycony obraz ma lepszą jakość (nie jest ,,ruszony''. Wadą zastosowania kartonowego pudła jest zmniejszenie jasności uzyskiwanych obrazów. Do rozjaśnienia obrazu użyto oświetlacza IR oraz lampy umieszczonej na kamerze, która oświetla oko badanej osoby. Taka lampa również tworzy odblask, lecz w tym wypadku odblask może być wykorzystany przez algorytm wykrywający źrenicę. Jest to możliwe, ponieważ z uwagi na umiejscowienie lampy odblask powstaje na środku źrenicy, dzięki czemu jest łatwo znaleźć ten środek. Obraz jest zapisywany w skali szarości.

\section{Wykrycie źrenicy}
\label{sec:wykrycieZrenicy}
Część algorytmu odpowiedzialna za wykrycie źrenicy ma za zadanie odnalezienie źrenicy na obrazie oraz opisanie kształtu źrenicy jako okrąg. Zadanie to jest realizowane z wykorzystaniem wcześniej wspomnianego odblasku pochodzącego od lampy, który powinien znajdować się na środku źrenicy. Odblask jest jednym z najjaśniejszych punktów na obrazie. W związku z tym wykrycie odblasku odbywa się z wykorzystaniem binaryzacji o progu 254. Po tej operacji na obrazie zostają tylko piksele najjaśniejsze. Niestety na wielu ujęciach odblask na źrenicy nie jest jedynym, który jest widoczny na obrazie. Pozostałe mogą znajdować się w różnych miejscach: na białku oka lub na skórze. Można ten fakt wykorzystać analizując otoczenie znalezionych odblasków. Interesujący nas odblask znajduje się w otoczeniu źrenicy, a więc jednego z najciemniejszych obszarów obrazu, otoczenie pozostałych odblasków jest o wiele jaśniejsze (skóra lub białko oka). Dlatego algorytm analizuje otoczenie każdego odblasku. Badanych jest 20 pikseli w odległości od 10 do 20 pikseli. Wybierany pierwszy odblask, w którego otoczeniu większość pikseli ma jasność poniżej 50. Po znalezieniu jednego piksela odblasku łatwo można znaleźć cały odblask. Jest to realizowane przez analizę otoczenia znalezionego piksela (w odległości nie większej niż 25 pikseli) i klasyfikowaniu ich jako element odblasku jeśli ich jasność jest większa niż 240. Po znalezieniu wszystkich pikseli należących do odblasku jego kształt jest aproksymowany kołem, którego środek jest przyjmowany jako środek źrenicy. Mając środek źrenicy można zastosować metodę ,,eksplodujących okręgów'' z patentu Dougmana.Szukane jest największe zmniejszenie się jasności pikseli na rozszerzającym się okręgu (z uwagi na to, że źrenica jest ciemna a tęczówka jest jaśniejsza). Znaleziony okrąg określa źrenicę.

\section{Wykrycie źrenicy}
\label{sec:wykrycieZrenicy}
