\chapter{Podsumowanie}
\label{cha:podsumowanie}

Można uznać, że realizacja systemu powiodła się. Poprawiono sposób działania systemu przedstawionego w pracy \cite{Gl11}. Głównymi usprawnieniami było pełne zautomatyzowanie procesu segmentacji źrenicy, rozbudowa bazy danych oraz rozbudowa interfejsu użytkownika powiązana ~z obsługą bazy danych. Również stanowisko do akwizycji zostało uproszczone, ponieważ nie jest już konieczna lampa do oświetlenia. Ponieważ zmianie uległ sposób akwizycji obrazu, co miało wpływ na wygląd pobieranych obrazów, algorytm segmentacji źrenicy musiał być stworzony od nowa.

~W dalszym rozwoju systemu początkowym punktem prac powinno być usprawnienie stanowiska. Należało by podjąć próby zmniejszenia jego rozmiaru oraz wyeliminować pudło. Dokonywanie ulepszeń algorytmu zalecane jest oo zakończeniu prac nad stanowiskiem, ponieważ każda zmiana sposobu rejestracji zdjęć może spowodować, że aktualny algorytm przestanie działać (dotyczy to głównie segmentacji).

I co dalej?

