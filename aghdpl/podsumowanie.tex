\chapter{Podsumowanie}
\label{cha:podsumowanie}

Biorąc pod uwagę wyniki przeprowadzonych testów można uznać, że realizacja systemu zakończyła się powodzeniem. Ulepszono sposób działania systemu przedstawionego w pracy \cite{Gl11}. Wśród głównych usprawnień można wyróżnić:
\begin{itemize}
\item pełne zautomatyzowanie procesu segmentacji źrenicy,
\item rozbudowę bazy danych,
\item rozbudowę interfejsu użytkownika powiązaną ~z obsługą zaawansowanej bazy danych.
\end{itemize}

Dodatkowo stanowisko do akwizycji obrazu zostało uproszczone, ponieważ nie jest już konieczne użycie lampy do oświetlenia sceny. ~W związku ze zmianą sposobu akwizycji obrazu, mającej wpływ na wygląd pobieranych obrazów, algorytm segmentacji źrenicy musiał być stworzony od podstaw. 

~W przypadku dalszego rozwoju systemu początkowym punktem prac powinno być ulepszenie stanowiska, w celu zapewnienia jego przenośności. Należałoby podjąć próby zmniejszenia jego rozmiaru, wyeliminowania kartonowego pudła otaczjącego kamerę i stojaka unieruchamiającego głowę. Dokonywanie ulepszeń algorytmu zalecane jest po zakończeniu prac nad stanowiskiem, ponieważ każda zmiana sposobu rejestracji zdjęć może spowodować niepoprawne działanie algorytmu (dotyczy to głównie etapu segmentacji).

Kolejnym możliwym usprawnieniem aplikacji może być dodanie funkcji identyfikacji na podstawie innych biometryk, takich jak odcisk palca lub kształt twarzy człowieka. ~W tym celu wystarczyłoby zaimplementować sposób pobierania, przetwarzania oraz porównywania danej biometryki, gdyż w bazie danych zostały przewidziane pola służące do przechowywania informacji dla alternatywnych metod identyfikacji.

Stworzony system realizuje stawiane mu wymagania. W związku z tym, może stanowić odmienną formę dobre fundamenty pod budowę systemu sprawdzającego obecność na zajęciach.
