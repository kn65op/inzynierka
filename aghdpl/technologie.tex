\chapter{Implementacja}
\label{cha:projekt}
~W tym rozdziale przedstawiono charakterystykę technologii oraz algorytmów użytych podczas implementacji rozwiązania. Część \ref{sec:przetwarzanieObrazow} opisuje szczegóły poprawy ~i wzbogacania systemu ~w części biometrycznej, podrozdział \ref{sec:czescBazodanowa} zawiera informacje na temat implementacji bazy danych programu ~i funkcjonalności aplikacji koniecznych do pełnego wykorzystania oferowanych opcji.

\section{Język programowania}
\label{sec:jezykProgramowania}
Wybranym językiem programowania jest C++. Wszelkie zalety tego języka, które miały wpływ na wybór zostały przedstawione ~w pracy \cite{Gl11}. Dzięki doświadczeniu w używaniu tego języka programowania można było szybciej oraz wydajniej zrealizować projekt.

\section{Biblioteka do przetwarzania obrazow}
\label{sec:bibliotekaObrazow}
Często używaną biblioteką używaną przy przetwarzaniu obrazów jest biblioteka OpenCV. Posiada dużą liczbę zaimplementowanych algorytmów oraz wykazuje szybką ich realizację. Także ~w tym projekcie została ona użyta. Główne czynniki wpływające na ten wybór są przedstawione ~w pracy \cite{Gl11}. Także doświadczenie ~w używaniu tej biblioteki miało wpływ. Dzięki temu można był szybciej oraz efektywniej zrealizować projekt.

\section{Biblioteka do interfejsu graficznego}
\label{sec:interfejsGraficzny}
Użytą biblioteką do tworzenia interfejsu graficznego jest biblioteka Qt. Jej zalety przedstawione są w pracy \cite{Gl11}. Znajomość tej biblioteki nie była duża, ale dzięki dobrej dokumentacji można było szybko poznać sposób działania oraz efektywnie wykorzystać możliwości biblioteki.

\section{Baza danych}
\label{sec:bazaDanych}

