\chapter{Wprowadzenie}
\label{cha:wprowadzenie}

\LaTeX~jest systemem sk�adu umo�liwiaj�cym tworzenie dowolnego typu dokument�w (w~szczeg�lno�ci naukowych i technicznych) o wysokiej jako�ci typograficznej (\cite{Dil00}, \cite{Lam92}). Wysoka jako�� sk�adu jest niezale�na od rozmiaru dokumentu -- zaczynaj�c od kr�tkich list�w do bardzo grubych ksi��ek. \LaTeX~automatyzuje wiele prac zwi�zanych ze sk�adaniem dokument�w np.: referencje, cytowania, generowanie spis�w (tre�li, rysunk�w, symboli itp.) itd.

\LaTeX~jest zestawem instrukcji umo�liwiaj�cych autorom sk�ad i wydruk ich prac na najwy�szym poziomie typograficznym. Do formatowania dokumentu \LaTeX~stosuje \TeX a (wymiawamy 'tech' -- greckie litery $\tau$, $\epsilon$, $\chi$). Korzystaj�c z~systemu sk�adu \LaTeX~mamy za zadanie przygotowa� jedynie tekst �r�d�owy, ca�y ci�ar sk�adania, formatowania dokumentu przejmuje na siebie system.

%---------------------------------------------------------------------------

\section{Cele pracy}
\label{sec:celePracy}

Celem poni�szej pracy jest zapoznanie student�w z systemem \LaTeX~w zakresie umo�liwiaj�cym im samodzielne, profesjonalne z�o�enie pracy dyplomowej w systemie \LaTeX.


%---------------------------------------------------------------------------

\section{Zawarto�� pracy}
\label{sec:zawartoscPracy}

W rodziale~\ref{cha:pierwszyDokument} przedstawiono podstawowe informacje dotycz�ce struktury dokument�w w \LaTeX u. Alvis~\cite{Alvis2011} jest j�zykiem 


















