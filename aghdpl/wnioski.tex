\chapter{Testy i wnioski}
\label{cha:testywnioski}
~W tym rozdziale przedstawiono wyniki testów stworzonej aplikacji oraz wnioski, które można wyciągnąć na podstawie przeprowadzonych testów.

Można pomyśleć o tytułach~~~~~~~~~~~~~~~~~~!!!!!!!!!!!!!!!!!!DASDSfabshjdbs jMożna pomyśleć o tytułach~~~~~~~~~~~~~~~~~~!!!!!!!!!!!!!!!!!!DASDSfabshjdbs j

\section{Testy systemu}
\label{sec:testy}
Celem implementacji systemu służącego do identyfikacji biometrycznej na podstawie tęczówki jest stwierdzenie, że tęczówki są identyczne wtedy ~i tylko wtedy, jeśli obie tęczówki należą do jednej osoby. W związku ~z tym taki system może pełnić dwa błędy:
\begin{itemize}
\item fałszywa identyfikacja - ~w sytuacji, gdy dwa analizowane zdjęcia przedstawiają tęczówki należące do różnych osób ~a system określi, że są identyczne,
\item fałszywe odrzucenie - ~w sytuacji, gdy dwa analizowane zdjęcia przedstawiają tęczówki należące do tej samej osoby ~a system określi je jako różne.
\end{itemize}
Sytuacja fałszywej identyfikacji jest groźniejsza, ~a także trudniejsza do eliminacji lub zmniejszenia prawdopodobieństwa. ~W celu zmniejszenia prawdopodobieństwa tego przypadku należałoby każdemu użytkownikowi robić dwa zdjęcia tęczówki ~i akceptować tylko w przypadku pozytywnej weryfikacji obu zdjęć. Fałszywe odrzucenie jest mniej groźne oraz łatwo można zrobić drugie zdjęcie tęczówki osoby, która została fałszywie odrzucona. ~W związku ~z tym celem było osiągnięcie jak najmniejszej liczby fałszywych identyfikacji, nawet kosztem zwiększenia liczby fałszywych odrzuceń.

Przy użyciu pierwszej kamery wyniki testów nie były obiecujące. Wartości odległości Hamminga dla różnych obrazów tęczówek niewiele się różniły od odległości wyliczonych dla obrazów tej samej tęczówki. Co więcej zdarzało się, że odległość dla różnych obrazów tęczówek była niższa niż dla obrazów tej samej. ~W związku ~z tym ustalenie granicy odległości Hamminga, która wskazuje na zdjęcia tej samej tęczówki stało się zadaniem niewykonalnym. Osiągnięte wyniki były ~w głównej mierze spowodowane małą rozdzielczością kamery oraz słabym oświetleniem tęczówki.

TUTAJ MOŻE TABELA Z WYNIKAMI?

~W związku ~z tym należało użyć kamery ~o większej rozdzielczości. Po zastosowaniu takiej kamery wyniki były ~o wiele lepsze, ponadto nie wystąpił problem ze zbyt małym oświetleniem tęczówki. Widać wyraźną różnicę pomiędzy odległościami Hamminga dla obrazów tej samej tęczówki oraz dla obrazów różnych tęczówek.

TUTAJ TABELA Z WYNIKAMI I OPIS TABELI.

\section{Wnioski}
\label{sec:wnioski}

Pomyślnie udało się zrealizować główne zadanie jakim było stworzenie aplikacji pozwalającej na identyfikację osób na podstawie tęczówki. System spełnia stawiane mu wymagania odnośnie poprawności rozpoznania osób. Nie jest to oczywiście system idealny, posiada kilka wad.

Głównym problemem jest pierwszy etap: akwizycja obrazu. Obraz uzyskany jest dość dobrej jakości, jednak sposób uzyskania obrazu (stanowisko, pozycja badanego). Utrudnia to ~w znacznym stopniu zastosowanie systemu ~w różnych lokalizacjach, aby móc skorzystać ~z niego ~w innym miejscu, poza przeniesieniem wymaganych elementów, należałoby również ustawić wszystko ~w odpowiednim układzie (kamera, podpórka pod głowę, pudło). Byłoby to czasochłonne oraz mogłoby się okazać, że nie ~w każdym pomieszczeniu można zestawić stanowisko. Dodatkowo ostrość ustawiana jest ręcznie co utrudnia korzystanie ~z urządzenia. ~W celu poprawy należało zainwestować ~w lepsze urządzenia: kamerę potrafiącą rejestrować tylko ~w podczerwieni (co pozwoliłoby na ograniczenie odblasków) oraz obiektyw ~z automatycznym ustawianiem ostrości.

Problemy również sprawiały powieki oraz rzęsy, które zasłaniają większą część oka podczas akwizycji. Powoduje to problemy przy segmentacji źrenicy (źrenica łączy się z rzęsami podczas binaryzacji) oraz przy tworzeniu kodu tęczówki (górna ~i dolna część tęczówki nie jest używana, ale zasłaniane są fragmenty ~z boku, które są używane). Badany, aby zostać poprawnie rozpoznanym musi rozszerzyć powieki. Głównie jest to widoczne przy górnej powiece ~i problem pojawia się zawsze, nawet przy obniżeniu kamery. Miało to na celu spowodować, że badany patrząc na kamerę będzie patrzył ~w dół, co spowoduje, że powieka będzie zasłaniać mniejszą część oka, ale nie zaobserwowanego takie zachowania powieki. ~W związku z tym należy ~w przyszłości podjąć próby rozwiązania tego problemu ~w inny sposób.

~W przypadku zmiany kamery mogłoby się okazać, że sposób znajdowania źrenicy ~i tęczówki wymagałby modyfikacji. Przy obecnie wykorzystywanych te algorytmy działają poprawnie oraz ~w pełni automatycznie.

Baza danych na chwilę obecną jest dobrze zaprojektowana. Podczas używania ~z uwagi na zmianę czynników zewnętrznych może się okazać, że potrzebne będą nowe pola lub niektóre okażą się niepotrzebne. ~Z uwagi na odpowiedni podział na tabele nie powinno to przysparzać większych kłopotów.




