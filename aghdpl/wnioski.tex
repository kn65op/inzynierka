\chapter{Testy i wnioski}
\label{cha:testywnioski}
~W tym rozdziale przedstawiono wyniki testów stworzonej aplikacji oraz wnioski, które można wyciągnąć po przeprowadzonych testach.

Można pomyśleć o tytułach~~~~~~~~~~~~~~~~~~!!!!!!!!!!!!!!!!!!DASDSfabshjdbs jMożna pomyśleć o tytułach~~~~~~~~~~~~~~~~~~!!!!!!!!!!!!!!!!!!DASDSfabshjdbs j

\section{Testy systemu}
\label{sec:testy}
Celem implementacji systemu służącego do identyfikacji biometrycznej na podstawie tęczówki oka jest stwierdzenie, że tęczówki są identyczne wtedy ~i tylko wtedy, jeśli obie tęczówki należą do jednej osoby. W związku ~z tym taki system może pełnić dwa błędy:
\begin{itemize}
\item fałszywa identyfikacja - ~w sytuacji, gdy dwa analizowane zdjęcia przedstawiają tęczówki należące do różnych osób ~a system określi, że są identyczne,
\item fałszywe odrzucenie - ~w sytuacji, gdy dwa analizowane zdjęcia przedstawiają tęczówki należące do tej samej osoby ~a system określi je jako różne.
\end{itemize}
Sytuacja fałszywej identyfikacji jest groźniejsza, ~a także trudniejsza do eliminacji lub zmniejszenia prawdopodobieństwa. 

\section{Wnioski}
\label{sec:wnioski}
