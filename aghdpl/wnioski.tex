\chapter{Testy i~wnioski}
\label{cha:testywnioski}
W~tym rozdziale przedstawiono wyniki testów stworzonej aplikacji oraz wnioski, które można wyciągnąć na podstawie przeprowadzonych testów.

\section{Testy systemu}
\label{sec:testy}
Podstawą implementacji systemu służącego do identyfikacji biometrycznej na podstawie tęczówki jest stwierdzenie, że tęczówki są identyczne wtedy i~tylko wtedy, jeśli obie należą do jednej osoby. W związku z~tym działanie takiego systemu może zakończyć się wystąpieniem dwóch rodzajów błędów:
\begin{itemize}
\item fałszywa identyfikacja - w~sytuacji, gdy dwa analizowane zdjęcia przedstawiają tęczówki należące do różnych osób, a~system określi, że są identyczne,
\item fałszywe odrzucenie - w~sytuacji, gdy dwa analizowane zdjęcia przedstawiają tęczówki należące do tej samej osoby, a~system określi je jako różne.
\end{itemize}
Sytuacja fałszywej identyfikacji jest groźniejsza, a~także trudniejsza do eliminacji lub zmniejszenia prawdopodobieństwa jej wystąpienia. W~celu ograniczenia częstości występowania tego przypadku można by dla każdego użytkownika przechować w~bazie dwa zdjęcia tęczówki i~identyfikować tylko w~przypadku pozytywnej weryfikacji dla obu zdjęć. Fałszywe odrzucenie jest mniej groźne w~skutkach oraz łatwo można pobrać drugie zdjęcie tęczówki osoby, która została fałszywie odrzucona. W~związku z~tym, celem było osiągnięcie jak najmniejszej liczby fałszywych identyfikacji, nawet kosztem zwiększenia liczby fałszywych odrzuceń.

Przy użyciu pierwszej kamery wyniki testów nie były obiecujące. Wartości odległości Hamminga dla różnych obrazów tęczówek niewiele się różniły od odległości wyliczonych dla obrazów tej samej tęczówki. Co więcej zdarzało się, że odległość dla obrazów różnych tęczówek była niższa niż dla obrazów przedstawiających tęczówkę jednej osoby. W~związku z~tym ustalenie granicy odległości Hamminga, dzięki której można by było poprawnie identyfikować tożsamość osoby, stało się zadaniem niewykonalnym. Osiągnięte wyniki były niezadowalające, a~głównymi czynnikami wpływającymi na ten stan były mała rozdzielczość kamery oraz słabe oświetlenie tęczówki.

Tabela \ref{tab:pierwsza} przedstawia wyniki testów dla pierwszej kamery. Przedstawione są na niej wyliczone wartości odległości Hamminga dla kodów tęczówek pochodzących od czterech osób (A, B, C, D), po trzy zdjęcia dla każdej osoby (numery zdjęć 1, 2 i~3). Widać wyraźnie, że uzyskane wartości nie pozwalają na poprawne rozpoznanie osób. Często występuje przypadek, że najmniejsze wyliczone wartości odległości Hamminga są osiągane dla obrazów przedstawiających różne tęczówki.

\begin{table}
\begin{center}
\caption{Tabela przedstawiająca fragment testów systemu dla obrazów pobranych pierwszą kamerą}
\label{tab:pierwsza}
\begin{tabular}{|c|c|c|c|c|c|c|c|c|c|c|c|c|c|c|c|c|c|c|}
\hline
 & A1 & A2 & A3 & B1 & B2 & B3 & C1 & C2 & C3 & D1 & D2 & D3\\ \hline
A1 & 0&0,201&0,212&0,213&0,205&0,215&0,227&0,235&0,213&0,233&0,229&0,231 \\ \hline
A2 & 0,201&0&0,220&0,221&0,228&0,240&0,233&0,249&0,220&0,245&0,232&0,237 \\ \hline
A3 & 0,212&0,214&0&0,228&0,231&0,229&0,245&0,240&0,234&0,232&0,241&0,242\\ \hline
B1 & 0,213&0,227&0,228&0&0,208&0,213&0,199&0,213&0,193&0,228&0,242&0,226\\ \hline
B2 & 0,205&0,228&0,231&0,200&0&0,188&0,216&0,228&0,209&0,228&0,232&0,244\\ \hline
B3 & 0,216&0,237&0,229&0,204&0,188&0&0,216&0,217&0,193&0,229&0,232&0,234\\ \hline
C1 & 0,228&0,235&0,242&0,199&0,219&0,217&0&0,189&0,200&0,234&0,243&0,238\\ \hline
C2 & 0,239&0,240&0,240&0,213&0,224&0,217&0,189&0&0,195&0,229&0,245&0,224\\ \hline
C3 & 0,214&0,234&0,234&0,193&0,209&0,193&0,194&0,195&0&0,234&0,233&0,236\\ \hline
D1 & 0,235&0,245&0,235&0,228&0,228&0,229&0,233&0,229&0,234&0&0,199&0,202\\ \hline
D2 & 0,233&0,232&0,243&0,242&0,242&0,232&0,243&0,239&0,233&0,199&0&0,200\\ \hline
D3 & 0,243&0,237&0,243&0,226&0,235&0,228&0,233&0,221&0,229&0,202&0,200&0\\ \hline
\end{tabular}
\end{center}
\end{table}

~W związku z~osiągniętymi wynikami zaprezentowanymi w~tabeli \ref{tab:pierwsza} należało użyć kamery o~większej rozdzielczości. Po zastosowaniu takiej kamery wyniki okazały się być o~wiele lepsze. Ponadto nie wystąpił problem ze zbyt małym oświetleniem tęczówki. Widać wyraźną różnicę pomiędzy odległościami Hamminga dla obrazów tej samej tęczówki oraz dla obrazów różnych tęczówek.

Tabele \ref{tab:druga1} i~\ref{tab:druga2} przedstawiają wynik testów systemu dla obrazów pobieranych za pomocą drugiej kamery. Przedstawione są na nich wartości obliczonych odległości Hamminga dla kodów tęczówek pochodzących od sześciu osób (A, B, C, D, E i~F), których kody tęczówek są już zapisane w~bazie (kolumny w~tabeli) oraz trzech osób (G, H, I), dla których nie ma kodów w~bazie i~identyfikacja nie powinna się powieść. W~ostatniej kolumnie przedstawiono wniosek z~porównań:
\begin{itemize}
\item rozpoznanie - dana osoba posiada zdjęcie w~porównywanej bazie i~została poprawnie rozpoznana,
\item nie rozpoznano - dana osoba nie posiada zdjęcia w~porównywanej bazie i~nie została rozpoznana,
\item błędnie rozpoznano - dana osoba nie posiada zdjęcia w~porównywanej bazie, ale została rozpoznana,
\item błędnie nie rozpoznano - dana osoba posiada zdjęcie w~porównywanej bazie, ale nie została rozpoznana.
\end{itemize}

Można zauważyć, że wystąpiły jedynie dwie pierwsze sytuacje z~przedstawionych powyżej. Oznacza to, że dla danych testowych system działa w~pełni poprawnie.

\begin{table}
\begin{center}
\caption{Tabela przedstawiająca fragment testów systemu dla obrazów pobranych drugą kamerą}
\label{tab:druga1}
\begin{tabular}{|c|c|c|c|c|c|c|c|c|c|c|c|c|c|c|c|c|c|c|l|}
\hline
 & A1 & A2 & A3 & B1 & B2 & B3 & C1 & C2 & C3 & wniosek\\ \hline
A1 & 0&0,114&0,121&0,194&0,196&0,195&0,212&0,213&0,208 & rozpoznano\\ \hline
A2 & 0,114&0&0,078&0,192&0,186&0,184&0,207&0,21&0,207 & rozpoznano\\ \hline
A3 & 0,121&0,078&0&0,198&0,194&0,189&0,207&0,213&0,208 & rozpoznano\\ \hline
B1 & 0,194&0,192&0,198&0&0,124&0,108&0,2&0,206&0,198 & rozpoznano\\ \hline
B2 & 0,196&0,186&0,194&0,124&0&0,065&0,191&0,203&0,202 & rozpoznano\\ \hline
B3 & 0,195&0,184&0,189&0,108&0,065&0&0,199&0,202&0,204 & rozpoznano\\ \hline
C1 & 0,212&0,207&0,207&0,2&0,191&0,199&0&0,121&0,145 & rozpoznano\\ \hline
C2 & 0,213&0,21&0,213&0,206&0,203&0,202&0,121&0&0,093 & rozpoznano\\ \hline
C3 & 0,208&0,207&0,208&0,198&0,202&0,204&0,145&0,093&0 & rozpoznano\\ \hline
D1 & 0,186&0,197&0,201&0,219&0,216&0,213&0,21&0,211&0,212 & nie rozpoznano\\ \hline
D2 & 0,186&0,194&0,197&0,212&0,208&0,208&0,211&0,214&0,216 & nie rozpoznano\\ \hline
D3 & 0,185&0,199&0,197&0,223&0,216&0,211&0,216&0,222&0,225 & nie rozpoznano\\ \hline
E1 & 0,186&0,2&0,198&0,216&0,205&0,206&0,215&0,218&0,208 & nie rozpoznano\\ \hline
E2 & 0,181&0,192&0,191&0,218&0,205&0,204&0,209&0,214&0,209 & nie rozpoznano\\ \hline
E3 & 0,182&0,198&0,196&0,213&0,195&0,196&0,214&0,218&0,202 & nie rozpoznano\\ \hline
F1 & 0,2&0,201&0,2&0,211&0,195&0,192&0,21&0,21&0,206 & nie rozpoznano\\ \hline
F2 & 0,206&0,219&0,222&0,23&0,218&0,222&0,225&0,227&0,224 & nie rozpoznano\\ \hline
F3 & 0,192&0,198&0,197&0,201&0,201&0,196&0,213&0,215&0,213 & nie rozpoznano\\ \hline
G1 & 0,193&0,208&0,207&0,216&0,202&0,197&0,216&0,218&0,229 & nie rozpoznano\\ \hline
G2 & 0,196&0,204&0,202&0,213&0,192&0,195&0,217&0,22&0,226 & nie rozpoznano\\ \hline
H1 & 0,19&0,2&0,207&0,211&0,206&0,217&0,217&0,218&0,207 & nie rozpoznano\\ \hline
H2 & 0,186&0,201&0,207&0,211&0,207&0,214&0,206&0,213&0,211 & nie rozpoznano\\ \hline
I1 & 0,188&0,211&0,209&0,214&0,205&0,209&0,215&0,219&0,218 & nie rozpoznano\\ \hline
I2 & 0,188&0,207&0,209&0,213&0,207&0,207&0,212&0,22&0,22 & nie rozpoznano\\ \hline
\end{tabular}
\end{center}
\end{table}

\begin{table}
\begin{center}
\caption{Tabela przedstawiająca fragment testów systemu dla obrazów pobranych drugą kamerą}
\label{tab:druga2}
\begin{tabular}{|c|c|c|c|c|c|c|c|c|c|c|c|c|c|c|c|c|c|c|l|}
\hline
 & D1 & D2 & D3 & E1 & E2 & E3 & F1 & F2 & F3 & wniosek\\ \hline
A1 & 0,186&0,186&0,185&0,186&0,181&0,182&0,2&0,206&0,192 & nie rozpoznano\\ \hline
A2 & 0,197&0,194&0,199&0,2&0,192&0,198&0,201&0,219&0,198 & nie rozpoznano\\ \hline
A3 & 0,201&0,197&0,197&0,198&0,191&0,196&0,2&0,222&0,197 & nie rozpoznano\\ \hline
B1 & 0,219&0,212&0,223&0,216&0,218&0,213&0,211&0,23&0,201 & nie rozpoznano\\ \hline
B2 & 0,216&0,208&0,216&0,205&0,205&0,195&0,195&0,218&0,201 & nie rozpoznano\\ \hline
B3 & 0,213&0,208&0,211&0,206&0,204&0,196&0,192&0,222&0,196 & nie rozpoznano\\ \hline
C1 & 0,21&0,211&0,216&0,215&0,209&0,214&0,21&0,225&0,213 & nie rozpoznano\\ \hline
C2 & 0,211&0,214&0,222&0,218&0,214&0,218&0,21&0,227&0,215 & nie rozpoznano\\ \hline
C3 & 0,212&0,216&0,225&0,208&0,209&0,202&0,206&0,224&0,213 & nie rozpoznano\\ \hline
D1 & 0&0,071&0,067&0,198&0,201&0,197&0,191&0,222&0,191 & rozpoznano\\ \hline
D2 & 0,071&0&0,06&0,189&0,189&0,186&0,189&0,227&0,193 & rozpoznano\\ \hline
D3 & 0,067&0,06&0&0,196&0,2&0,191&0,192&0,225&0,187 & rozpoznano\\ \hline
E1 & 0,198&0,189&0,196&0&0,049&0,041&0,205&0,195&0,183 & rozpoznano\\ \hline
E2 & 0,201&0,189&0,2&0,049&0&0,054&0,211&0,202&0,189 & rozpoznano\\ \hline
E3 & 0,197&0,186&0,191&0,041&0,054&0&0,208&0,193&0,19 & rozpoznano\\ \hline
F1 & 0,191&0,189&0,192&0,205&0,211&0,208&0&0.121&0,175 & rozpoznano\\ \hline
F2 & 0,222&0,227&0,225&0,195&0,202&0,193&0.121&0&0,162 & rozpoznano\\ \hline
F3 & 0,191&0,193&0,187&0,183&0,189&0,19&0,175&0,162&0 & rozpoznano\\ \hline
G1 & 0,195&0,192&0,194&0,203&0,208&0,2&0,203&0,223&0,203 & nie rozpoznano\\ \hline
G2 & 0,189&0,187&0,194&0,201&0,2&0,196&0,2&0,224&0,196 & nie rozpoznano\\ \hline
H1 & 0,207&0,208&0,214&0,193&0,192&0,2&0,211&0,208&0,208 & nie rozpoznano\\ \hline
H2 & 0,198&0,202&0,207&0,192&0,191&0,192&0,208&0,206&0,206 & nie rozpoznano\\ \hline
I1 & 0,188&0,185&0,19&0,184&0,186&0,186&0,189&0,201&0,181 & nie rozpoznano\\ \hline
I2 & 0,196&0,194&0,196&0,185&0,18&0,18&0,191&0,198&0,191 & nie rozpoznano\\ \hline
\end{tabular}
\end{center}
\end{table}

\section{Wnioski}
\label{sec:wnioski}

Pomyślnie udało się zrealizować główne zadanie jakim było stworzenie aplikacji pozwalającej na identyfikację osób na podstawie tęczówki. System spełnia stawiane mu wymagania odnośnie poprawności rozpoznania osób. Nie jest to oczywiście system idealny. Wciąż istnieją fragmenty systemu, których poprawa znacznie ulepszy system.

Głównym problemem jest realizacja pierwszego etapu działania systemu: akwizycja obrazu. Uzyskany obraz jest dobrej jakości, jednak sposób jego  rejestracji (stanowisko, pozycja badanego) nie jest w~pełni zadowalający. Rozmiar stanowiska oraz jego sposób budowy utrudnia w~znacznym stopniu przenośność systemu. W celu skorzystania z~systemu w~dowolnym miejscu, poza przeniesieniem wymaganych elementów, należałoby również ustawić elementy stanowiska w~odpowiednim układzie (kamera, stojak pod głowę, kartonowe pudło). Byłoby to czasochłonne oraz istnieje możliwość, że nie w~każdym pomieszczeniu zestawienie stanowiska jest wykonalne. Dodatkowo ostrość ustawiana jest ręcznie, co utrudnia korzystanie z~urządzenia. W~celu poprawy należałoby zainwestować w~urządzenia lepiej spełniające swoje zadania: kamerę lub filtr, który pozwoliłby na zniwelowanie wpływu odblasków oraz obiektyw z~automatycznym ustawianiem ostrości lub z~możliwym bezdotykowym sterowaniem.

Kolejnym problemem są powieki oraz rzęsy, które zasłaniają większą część oka podczas akwizycji. ~Z tego powodu występują błędy podczas segmentacji źrenicy (źrenica łączy się z~rzęsami podczas binaryzacji) oraz przy tworzeniu kodu tęczówki (górna i~dolna część tęczówki oka nie jest używana, ale zdarza się, że zasłaniane są inne fragmenty, z~których jest tworzony kod). Badany, aby zostać poprawnie rozpoznanym, musi rozszerzyć powieki. Głównie jest to widoczne w~górnej części oka i~problem pojawia się zawsze, nawet przy zmianie ustawienia kamery względem oka. W~związku z~tym należy w~przyszłości podjąć próby rozwiązania tego problemu.

~W przypadku zmiany kamery mogłoby się okazać, że sposób znajdowania źrenicy i~tęczówki wymagałby modyfikacji. Przy obecnie wykorzystywanych urządzeniach algorytmy te działają poprawnie oraz w~pełni automatycznie.

Podczas rejestracji użytkownika należy zwrócić uwagę na to, by zdjęcie, z~którego będzie tworzony kod tęczówki zapisany w~bazie było odpowiednie. W~przypadku błędnego wprowadzenia kodu tęczówki późniejsza identyfikacja może okazać się niemożliwa. W sytuacji, gdy system będzie miał za zadanie zidentyfikować użytkownika zapisanego w~bazie, będzie można, w~przypadku zdjęcia dla którego segmentacja tęczówki nie powiedzie się, zrobić kolejne zdjęcie, które będzie odpowiednie. W~związku z~tym w~stworzonej aplikacji został zaimplementowany proces sprawdzania jakości zdjęcia i~segmentacji tęczówki. Wymaga on interakcji z~użytkownikiem, ale na etapie dodawania nowych osób do bazy, kiedy taka interakcja jest konieczna do wprowadzenia danych.

Baza danych na chwilę obecną jest poprawnie zaprojektowana. Podczas jej używania, z~uwagi na zmianę czynników zewnętrznych może się okazać, że konieczne będzie dodawanie nowych pól lub usunięcie niepotrzebnych. Dzięki odpowiedniemu podziałowi na tabele nie powinno przysparzać to większych problemów.

