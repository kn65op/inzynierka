\documentclass{beamer}
\usepackage[utf8]{inputenc}
\usepackage[T1]{fontenc}
\usepackage[polish]{babel}
\usepackage{graphicx}
\usepackage{times}

\usetheme{AGH}

\title[Biometryczny System Kontroli Dostępu]{Biometryczny System Kontroli Dostępu}

\author[T. Drzewiecki, J. Gajda]{Tomasz Drzewiecki, Joanna Gajda}

\date[2012]{9.01.2012}

\institute[AGH-EAIE]
{Wydział Elektrotechniki, Automatyki, Informatyki i Elektroniki\\ 
Katedra Automatyki
}

\setbeamertemplate{itemize item}{$\maltese$}

\begin{document}

{
%\usebackgroundtemplate{\includegraphics[width=\paperwidth]{titlepage}} % wersja angielska
\usebackgroundtemplate{\includegraphics[width=\paperwidth]{titlepagepl}} % wersja polska
 \begin{frame}
   \titlepage
 \end{frame}
}

%---------------------------------------------------------------------------

\begin{frame}
\frametitle{Tematyka projetku}

\begin{block}{Cel projektu}
Celem naszego projektu było stworzenie/rozbudowa systemu do identyfikacji osób na podstawie tęczówki oka.
\end{block}

\begin{figure}
\begin{center}
\includegraphics[scale=0.04]{teczowka.jpg}
\caption{Tęczówka człowieka}
\label{fig:teczowka}
\end{center}
\end{figure}

\end{frame}

%---------------------------------------------------------------------------

\begin{frame}
\frametitle{pusty slajd}

\end{frame}

%---------------------------------------------------------------------------

\begin{frame}
\frametitle{Koniec}

\begin{block}{}
Dziękujemy za uwagę.\\
Pytania?
\end{block}


\end{frame}

%---------------------------------------------------------------------------


\end{document}

