\documentclass{beamer}
\usepackage[utf8]{inputenc}
\usepackage[T1]{fontenc}
\usepackage[polish]{babel}
\usepackage{graphicx}
\usepackage{times}

\usetheme{AGH}

\title[Biometryczny System Kontroli Dostępu]{Biometryczny System Kontroli Dostępu}

\author[T. Drzewiecki, J. Gajda]{Tomasz Drzewiecki, Joanna Gajda}

\date[2012]{17.01.2012}

\institute[AGH-UST]
{Wydział Elektrotechniki, Automatyki, Informatyki i Elektroniki\\ 
Katedra Automatyki
}

\setbeamertemplate{itemize item}{$\maltese$}

\begin{document}

{
%\usebackgroundtemplate{\includegraphics[width=\paperwidth]{titlepage}} % wersja angielska
\usebackgroundtemplate{\includegraphics[width=\paperwidth]{titlepagepl}} % wersja polska
 \begin{frame}
   \titlepage
 \end{frame}
}

%---------------------------------------------------------------------------

\begin{frame}
\frametitle{Tematyka projektu}

\begin{block}{Cel projektu}
Celem naszego projektu było stworzenie/rozbudowa systemu do identyfikacji osób na podstawie tęczówki oka.
\end{block}

\begin{figure}
\begin{center}
\includegraphics[scale=0.04]{teczowka.jpg}
\caption{Tęczówka człowieka}
\end{center}
\end{figure}

\end{frame}

%---------------------------------------------------------------------------

\begin{frame}
\frametitle{Podział systemu}

Tworzony przez nas system składa się z trzech głównych części:
\begin{columns}[t]
\column{0.3\textwidth}
\begin{block}{Część sprzętowa}
Moduł odpowiedzialny za pobranie obrazu tęczówki od badanych osób. Najważniejszym elementem jest kamera oraz zapewnienie odpowiednich warunków.
\end{block}
\column{0.3\textwidth}
\begin{block}{Cześć biometryczna}
Część odpowiedzialna na zamianę obrazu tęczówki na jej kod oraz późniejsze porównywanie utworzonych kodów.
\end{block}
\column{0.3\textwidth}
\begin{block}{Część bazodanowa}
Część odpowiedzialna za przechowywanie danych osób wprowadzanych do systemu.
\end{block}
\end{columns} 

\end{frame}

%---------------------------------------------------------------------------

\begin{frame}
\frametitle{Konstrukcja stanowiska}
Stanowisko do pobierania obrazów tęczówki musi spełniać odpowiednie wymagania. Im lepsze będą pobierane obrazy, tym łatwiejsza będzie późniejsza segmentacja. Główne założenia:
\begin{itemize}
\item Likwidacja odblasków pochodzących z oświetlenia (naturalnego oraz sztucznego),
\item Odpowiednia jasność obrazu,
\item Wysoka jakość obrazu,
\item Odpowiednia wielkość tęczówki na obrazie.
\end{itemize}
\end{frame}
%---------------------------------------------------------------------------

\begin{frame}
\frametitle{Konstrukcja stanowiska}
Udało się skonstruować stanowisko spełniające wymagania:
\begin{itemize}
\item Odblaski usunięto poprzez przesłonięcie źródeł światła (za pomocą kartonowego pudła),
\item Odpowiednią jasność zapewnia oświetlacz IR,
\item Wysoką jakość obrazu zapewnia odpowiednia kamera, która potrafi rejestrować obrazy w dużej rozdzielczości,
\item Odpowiednią wielkość tęczówki otrzymujemy stosując przybliżenie za pomocą obiektywu.
\end{itemize}
\end{frame}

%---------------------------------------------------------------------------

\begin{frame}
\frametitle{Stworzone stanowisko}
\begin{columns}
	\column{0.5\textwidth}
		\begin{figure}
		\begin{center}
		\includegraphics[scale=0.05]{stanowisko1.jpg}
%		\caption{Obraz przed czyszczeniem brzegów}
		\end{center}
		\end{figure}
	\column{0.5\textwidth}
		\begin{figure}
		\begin{center}
		\includegraphics[scale=0.05]{stanowisko2.jpg}
%		\caption{Obraz po czyszczenu brzegów}
		\end{center}
		\end{figure}
\end{columns}\end{frame}

%---------------------------------------------------------------------------

\begin{frame}
\frametitle{Algorytm do tworzenia kodu tęczówki}
\begin{figure}
\begin{center}
\includegraphics[scale=0.4]{schemat.jpg}
\end{center}
\end{figure}
\end{frame}

%---------------------------------------------------------------------------

\begin{frame}
\frametitle{Wykrycie źrenicy}
Celem tej części algorytmu jest opisanie źrenicy przy pomocy sparametryzowanego okręgu, czyli znalezienie środka okręgu na obrazie oraz jego promienia. Punktem startowym jest pobrany obraz z kamery.
\begin{figure}
\begin{center}
\includegraphics[scale=0.13]{szarosc.jpg}
\end{center}
\end{figure}
\end{frame}

%---------------------------------------------------------------------------

\begin{frame}
\frametitle{Wykrycie źrenicy}
Następnie wyszukiwany jest odblask widoczny na źrenicy, pochodzący od oświetlacza IR. Poszukiwany jest jasny punkt w otoczeniu ciemnych. Po znalezieniu odblasku do dalszych przekształceń wybierany jest fragment obrazu w jego otoczeniu (kwadrat o boku 440 pikseli z odblaskiem w środku).
\end{frame}

%---------------------------------------------------------------------------

\begin{frame}
\frametitle{Wykrycie źrenicy}
Następnie w wybranym fragmencie obrazu stosujemy operację binaryzacji w celu wyznaczenia źrenicy z dwoma progami: mniejsze od 60 w celu znalezienia pikseli źrenicy oraz większe od 254 w celu znalezienie pikseli odblasku.
\begin{figure}
\begin{center}
\includegraphics[scale=0.25]{bin.jpg}
\end{center}
\end{figure}
\end{frame}

%---------------------------------------------------------------------------

\begin{frame}
\frametitle{Wykrycie źrenicy}
Kolejno stosowane są operacje zamknięcia i dwukrotnego otwarcia (za drugim razem ze zwiększonym obiektem strukturalnym) w celu otrzymania koła określającego źrenicę.
\begin{columns}
	\column{0.3\textwidth}
		\begin{figure}
		\begin{center}
		\includegraphics[scale=0.25]{zamkniecie.jpg}
		\end{center}
		\end{figure}
	\column{0.3\textwidth}
		\begin{figure}
		\begin{center}
		\includegraphics[scale=0.25]{otwarcie1.jpg}
		\end{center}
		\end{figure}
	\column{0.3\textwidth}
		\begin{figure}
		\begin{center}
		\includegraphics[scale=0.25]{otwarcie2.jpg}
		\end{center}
		\end{figure}
\end{columns}
\end{frame}

%---------------------------------------------------------------------------

\begin{frame}
\frametitle{Wykrycie źrenicy}
Ponieważ na niektórych obrazach po wcześniej wymienionych operacjach pozostaje biały obszar w miejscu, gdzie na obrazie występują rzęsy oraz obszar ten obszar zawsze jest styczny z brzegiem fragmentu obrazu, stosowany jest algorytm czyszczenia brzegów.
\begin{columns}
	\column{0.5\textwidth}
		\begin{figure}
		\begin{center}
		\includegraphics[scale=0.25]{roi1.jpg}
%		\caption{Obraz przed czyszczeniem brzegów}
		\end{center}
		\end{figure}
	\column{0.5\textwidth}
		\begin{figure}
		\begin{center}
		\includegraphics[scale=0.25]{roi2.jpg}
%		\caption{Obraz po czyszczenu brzegów}
		\end{center}
		\end{figure}
\end{columns}
\end{frame}

%---------------------------------------------------------------------------

%\begin{frame}
%\frametitle{Wykrycie tęczówki}
%Sposób wykrywania tęczówki jest zgodny z patentem Daugmana oraz z pracą z poprzedniego roku. Po wykonaniu algorytmu widzimy obszar na zdjęciu, w którym znajduje się tęczówka.
%\begin{figure}
%\begin{center}
%\includegraphics[scale=0.25]{teczowka_nasza.jpg}
%\end{center}
%\end{figure}
%\end{frame}

%---------------------------------------------------------------------------

\begin{frame}
\frametitle{Wykrycie tęczówki}
Sposób wykrywania tęczówki jest zgodny z patentem Daugmana; został zastosowany również w zeszłorocznej pracy. Po wykonaniu algorytmu widzimy obszar na zdjęciu, w którym znajduje się tęczówka.
\begin{figure}
\begin{center}
\includegraphics[scale=0.25]{teczowka_nasza.jpg}
\end{center}
\end{figure}
\end{frame}

%---------------------------------------------------------------------------

\begin{frame}
\frametitle{Tworzenie kodu tęczówki}
Kod tęczówki jest tworzony w następujący sposób:
\begin{itemize}
\item Wybierana jest odpowiednia liczba obszarów, tak, aby powstał kod o długości 2048 bitów,
\item Każdy z obszarów jest filtrowany filtrami Gabora (jest ich 8, każdy o innym kierunku),
\item Sumowane są otrzymane liczby zespolone,
\item Koduje się sumy, osobno część rzeczywistą oraz urojoną: 1 dla sumy danej części większej od 0, 0 dla sumy danej części mniejszej lub równej 0.
\end{itemize}
\end{frame}

%---------------------------------------------------------------------------

\begin{frame}
\frametitle{Porównywanie kodów}
Kody porównuje się licząc ile jest różnych pikseli dla dwóch kodów w stosunku do długości kodu. Używany jest wzór:
$$ H = \frac{\sum XOR(A,B)}{2048} $$
gdzie:
$A, B$ - porównywane kody tęczówek, każdy o długości 2048 bitów.\\
\end{frame}

%---------------------------------------------------------------------------

\begin{frame}
\frametitle{Wynik porównania}
Uznaje się, że dwa zdjęcia są obrazami tej samej tęczówki w przypadku, gdy wyliczona odległość Hamminga ma wartość poniżej 0.18. Jeśli jest równa lub większa to uznaje się, że obrazy przedstawiają różne tęczówki.
\end{frame}

%---------------------------------------------------------------------------
\begin{frame}
\frametitle{Baza danych programu}
W celu poszerzenia funkcjonalności aplikacji, postanowiono rozbudować istniejącą bazę danych systemu. Schemat przebudowanej bazy wygląda następująco:

\begin{figure}
\includegraphics[scale=0.3]{diagram.png}
\end{figure}
\end{frame}

\begin{frame}
\frametitle{Podstawowy interfejs aplikacji}
W podstawowej wersji programu obecny był wyłącznie podstawowy interfejs użytkownika, oferujący nieliczne funkcje wprowadzania informacji.
\begin{figure}
\includegraphics[scale=0.2]{oknoGlowne.jpg}
\end{figure}
\end{frame}

\begin{frame}
\frametitle{Rozszerzony interfejs aplikacji}
Rozbudowa bazy danych wiązała się z poszerzeniem interfejsu programu. Dodano funkcjonalności i zakładki:


\begin{itemize}
\item Zaawansowane opcje dodawania - zakładki: Dodaj kierunek, Dodaj przedmiot, Dodaj temat, Dodaj grupę, Dodaj zajęcia
\item Generowanie listy obecności
\end{itemize}
\end{frame}

\begin{frame}
\frametitle{Okno główne aplikacji}
Po przebudowie aplikacji, okno główne przyjęło postać:

\end{frame}

\begin{frame}
\frametitle{Zakładka Dodaj kierunek}
Pierwszą zakładką, w której możliwe jest uzupełnienie danych, jest Dodaj kierunek. Pozwala ona na dodanie specjalności o podanej nazwie i wydziale.

\end{frame}

\begin{frame}
\frametitle{Zakładka Dodaj przedmiot}
W zakładce drugiej możliwe jest dodanie przedmiotu o określonej nazwie, odbywającym się na konkretnym kierunku i roku studiów:

\end{frame}

\begin{frame}
\frametitle{Zakładka Dodaj temat}
W kolejnej zakładce można w ramach przedmiotu dodać realizowane tematy:

\end{frame}

\begin{frame}
\frametitle{Zakładka Dodaj grupę}
Zakładka Dodaj grupę pozwala na utworzenie grup studentów mających zajęcia w określone dni tygodnia i danych godzinach:

\end{frame}

\begin{frame}
\frametitle{Zakładka Dodaj zajęcia}
W zakładce Dodaj zajęcia możliwe jest dodanie zajęć z przedmiotu, wraz z podaniem terminu ich odbycia się oraz grupy i realizowanego tematu:

\end{frame}

\begin{frame}
\frametitle{Iplementacja}
\begin{itemize}
\item Język C++.
\item Biblioteka do przetwarzania obrazów: OpenCV.
\item Biblioteka do tworzenia interfejsu graficznego: Qt.
\item System zarządzania bazami danych: SQLite.
\end{itemize}
\end{frame}

%---------------------------------------------------------------------------


\begin{frame}
\frametitle{Wyniki i wnioski}
\begin{itemize}
\item Udało się skonstruować system do identyfikacji na podstawie tęczówki.
\item Najważniejsze jest uzyskanie dobrego obrazu, z jak najmniejszą liczbą zakłóceń.
\item Stworzone stanowisko nie jest przenośne.
\item Tęczówka na zdjęciu musi być duża, potrzeba jest dużej liczby szczegółów tęczówki.
\item Najtrudniejszym etapem podczas tworzenia kodu tęczowki jest segmentacja źrenicy.
\end{itemize}
\end{frame}

\begin{frame}
\frametitle{Koniec}

\begin{block}{}
Dziękujemy za uwagę.\\
Pytania?
\end{block}

\end{frame}

%---------------------------------------------------------------------------


\end{document}

